\subsection	{The Wiener Measure}

\begin	{definition}
A \term{quasi-Brownian motion} is a process~$B_t$, \( t \ge 0 \)
for which, for \( s < t \), the increment \( B_t - B_s \)
is normally distributed with mean~$0$ and variance \( t - s \),
independently of~$B_r$, \( r \le s \).
\end	{definition}

\begin	{lemma}
If $B_t$ is quasi-Brownian, then so are
\begin	{itemize}
\item	$B_{c^2 t}/c$, \( c \ne 0 \),
\item	\( B_t - B_{t_0} \), \( t_0 \ge 0 \).
\end	{itemize}
\end	{lemma}

\begin	{theorem}
Any quasi-Brownian motion admits a locally $\gamma$\dashHolder\ modification
for every \( \gamma < 1/2 \).
\end	{theorem}
\begin	{proof}
We can apply the Kolmogorov continuity theorem, %TODO reference
since normal variables have finite moments of any order and \[
	\Pr\bigl( \abs{B_t - B_s}^{2p} \bigr)
	=	\abs{t-s}^p \Pr(\abs{B_1}^{2p})
	\text.
	\qedhere
\]
\end	{proof}
%TODO Wiener measure
