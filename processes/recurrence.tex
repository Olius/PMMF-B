\section	{0--1~Law, Stopping Times, and Recurrence of~BM}

\setcounter	{block}	{27}
\begin	{theorem}[0--1~law]
Let $B_t$ be continous Brownian motion.
Set \( \mathcal F_{[a,b)} = \sgen(\, B_t - B_a \colon a \le t < b \,) \),
and \( \mathcal F_{0^+} = \bigwedge_{\epsilon>0} \mathcal F_{[0,\epsilon)} \).
Then \( \mathcal F_{0^+} \perp \mathcal F_{[0,\infty)} \).
\end	{theorem}
\begin	{proof}
\( \mathcal F_{0^+} \perp \mathcal F_{[\epsilon,\infty)} \)
for every \( \epsilon > 0 \),
so \( \mathcal F_{0^+}
\perp \bigvee_{\epsilon>0} \mathcal F_{[\epsilon,\infty)} \),
but by continuity \( \bigvee_{\epsilon>0} \mathcal F_{[\epsilon,\infty)}
= \mathcal F_{[0,\infty)} \).
\end	{proof}

\setcounter	{block}	{29}
\begin	{remark}
Suppose $X_t$ is a continuous stochastic process
on a topological space~$S$
in which every open \( U \subseteq S \) admits closed \( F_n \subseteq U \)
with \( U = \bigcup_n F_n^\circ \)
(for example a metric space).
The~$F_n$ can be assumed increasing.
Then for closed \( F \subseteq S \)
the entry time \( T_F = \min\{\, t \colon X_t \in F \,\} \)
is a stopping time:
\(	\{ T_F > t \}
=	\{ X_{[0,t]} \subseteq \co F = \bigcup_n F_n^\circ \}
=	\bigcup_n \{ X_{[0,t]} \subseteq F_n \}
=	\bigcup_n \bigcap_{0\le q\le t} \{ X_q \in F_n \}
\).
Alternatively, by \autoref{thm:co-subbasic-pointwise-measurable},
\( \{T_F > t\} = \{X_{[0,t]} \subseteq \co F\}
\in \sgen(\, X_t \colon t \ge 0 \,) \).
\end	{remark}

%TODO fix numbering
\begin	{proposition}
Almost surely \( \lim_{x\to0} T_x = 0 \).
Furthermore, \( T_x < \infty \) for all \( x \in \R \)
with probability~$1$.
\end	{proposition}
\begin	{proof}
\( T_x \to 0 \) if and only if $B_t$ changes sign
on every interval~$[0,\epsilon]$.
This is an event in~$\mathcal F_{0^+}$,
but \( 2\Pr(B_{[0,\epsilon]} \ge 0) \le 1 \).
Now suppose $B$ not surjective on~$\R$.
Then \( \pm B \) is bounded from below,
or equivalently by compactness,
there exist \( c \le 0 \le a \) such that \( B_{[a,\infty)} \ge c \).
But
\begin	{align*}
	\Pr\Bigl( \bigcup_{c \le 0 \le a} \{ B_{[a,\infty)} \ge c \} \Bigr)
	&=	\lim_{c\to-\infty} \lim_{a\to\infty}
			\Pr( B_{[1,\infty)} \ge c/\!\sqrt a ) \\
	&=	\lim_{c\to-\infty}
			\Pr( B_{[1,\infty)} \ge 0 ) \\
	&\le	\Pr(\,\text{\( B \ge 0 \) eventually}\,) \text,
\end	{align*}
and \( 2\Pr(\, B \ge 0 \) eventually\(\,) \le 1 \).
\end	{proof}

\begin	{corollary}
Almost surely, $B$ visits every \( x \in \R \) infinitely many times.
\end	{corollary}

\begin	{theorem}
\label	{thm:nowhere-holder}
	Brownian motion is almost surely nowhere locally $\alpha$\dashHolder\
	for any \( \alpha > 1/2 \).
\end	{theorem}
\begin	{proof}
Suppose $B_t$ is right $\alpha$\dashHolder\ in \( t = t_0 \).
Then by continuity and compactness, for some number~$M$, \[
	\text{\( \abs{B_{t_0+h} - B_{t_0}} \le Mh^\alpha \)
		when \(0 \le h \le 1 \).}
\]
Hence for any large enough~$n$,
whenever \( k-1 \le 2^n t_0 \le k \) and \( 1 \le j \le m \),
\begin	{align*}
	\abs{ B_{(k+j)/2^n} - B_{(k+j-1)/2^n} }
	&\le	\abs{ B_{(k+j)/2^n} - B_{t_0} }
		+ \abs { B_{(k+j-1)/2^n} - B_{t_0} } \\
	&\le	\frac {M\bigl((j+1)^\alpha+j^\alpha\bigr)} {2^{\alpha n}} \\
	&\le	\frac {2M(m+1)^\alpha} {2^{\alpha n}}
	\text.
\end	{align*}
It suffices to show this is has probability zero
for \( 0 \le t_0 \le 1 \).
Choose any \( C \ge 0 \), and let $\Omega_{nk}$ be the event \[
	\abs{ B_{2^{-n}(k+j)} - B_{2^{-n}(k+j-1)} } \le
		\frac C {2^{\alpha n}}
	\quad \text{for \( j = 1 \), \dots,~$m$.}
\]
By independence of increments, rescaling,
and that the normal density is~\( \le 1/2 \),
\begin	{align*}
	\Pr(\Omega_{nk}) &\le \Pr\bigl(
		\abs{B_1} \le \sqrt{2^n} C / 2^{\alpha n} \bigr)^m \\
	&\le	\frac {C^m} {2^{nm(\alpha-1/2)}}
	\text.
\end	{align*}
Therefore if we take \( m > (\alpha-1/2)^{-1} \), \[
	\Pr\Bigl( \bigcup_{k=0}^{2^n-m} \Omega_{nk} \Bigr)
	\le 2^{n - nm(\alpha-1/2)} C^m
\] is summable over~$n$,
and by the Borel--Cantelli Lemma,
$\bigcup_k \Omega_{nk}$ almost never occurs for arbitrarily large~$n$.
\end	{proof}

\begin	{theorem}
Over all partitions \( 0 = t_0 \le\dots\le t_n = t \), \[
	\text{\( \sum_{k<n} \abs{\Delta_k B_t}^2 \to t \) in~$\Ell^2$}
	\quad\text{as}\quad
	\max_{k<n} \Delta_k t \to 0
	\text.
\]
\end	{theorem}
\begin	{proof}
Since \( \abs{\Delta_k B_t}^2 - \Delta_k t \) are independent with mean~$0$,
\begin	{align*}
	\Pr\biggl[
		\Bigl( \sum_{k<n}
			\bigl( \abs{\Delta_k B_t}^2 - \Delta_k t \bigr) \Bigr)^2
	\biggr]
	&=	\sum_{k<n} \Pr\Bigl[
		\bigl( \abs{\Delta_k B_t}^2 - \Delta_k t \bigr)^2
	\Bigr] \\
\intertext
	{which, since $\Delta_k B_t$ is distributed
		like~$B_1\sqrt{\Delta_kt}$,}
	&=	\sum_{k<n} (\Delta_kt)^2
		\Pr\bigl[ \bigl(\abs{B_1}^2 - 1\bigr)^2 \bigr] \\
	&\le	t \Pr\bigl[ \abs{B_1}^4 - 1 \bigr] \max_{k<n} \Delta_kt \\
	&\to 0
	\text.
	\qedhere
\end	{align*}
\end	{proof}

\begin	{corollary}
Brownian motion is almost never anywhere of finite variation.
\end	{corollary}
\begin	{proof}
Convergence in~$\Ell^2$ implies almost sure convergence along a subsequence,
but quadratic variation on a given partition
is bounded by the partition width times simple variation.
\end	{proof}
