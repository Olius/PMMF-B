\subsection	{The Reflection Principle}

\begin	{theorem}
Let $T$ be an almost surely finite stopping time adapted to Brownian Motion.
Then \( B^{(T)}_t = B_{T+t} - B_T \) is Brownian Motion,
and is independent of~$\fil_T$.
\end	{theorem}
\begin	{proof}
Recall that if $V$ is locally compact Hausdorff,
then evaluation \( C(U,V) \times C(V,W) \to C(U,W) \) is continuous
when all mapping spaces are equipped with the compact-open topology.
Here, $T$ and \( T + t \) as functions of~$t$
can be viewed as random elements of~$C\bigl([0,\infty),[0,\infty)\bigr)$.
By second countability and local compactness then,
continuous evaluation \[
	C\bigl([0,\infty),[0,\infty)\bigr) \times
		C\bigl([0,\infty),\R)\bigr)
	\to C\bigl([0,\infty),\R\bigr)
\] is measurable over the product on the left.
Similar reasoning shows that continuous algebraic operations
applied to random continuous processes
yield further measurable continuous processes.
Therefore $B^{(T)}_t$ is measurable as the difference
of two compositions of random continuous processes.

To find the distribution of~$B^{(T)}_t$,
we express~$T$ as the decreasing limit of discrete stopping times
\( T_n = \ceil{Tn}/n \),
and show the theorem holds for discrete stopping times.
Then \( T_n \to T \) means that \( T_n + t \to T + t \)
in~$C\bigl([0,\infty),[0,\infty)\bigr)$,
so by continuity of composition \[
	B^{(T)}_t = \lim_{n\to\infty} B^{(T_n)}_t
	\quad \text{in~\( C[0,\infty) \).}
\]
Supposing the thesis holds for~$T_n$,
since each~$B^{(T_n)}_t$ is independent of~\( \fil_{T_n} \supseteq \fil_T \),
we obtain independence of~$B^{(T)}$ from~$\fil_T$.
In addition, its law must be the weak limit of the laws of the~$B^{(T_n)}$,
and since they are all Brownian Motion, so is~$B^{(T)}$.

Suppose now that $T$ is discrete,
\( \Pr\bigl(T\in\{t_1,t_2,\dotsc\}\bigr) = 1 \).
Then for any event \( A \subseteq C[0,\infty) \) and \( E \in \fil_T \),
in particular when \( \Pr(E) = 1 \),
\begin	{align*}
	\Pr\bigl(E\cap\{B^{(T)}\in A\}\bigr)
	&=	\sum_{n\ge1} \Pr\bigl(
			E \cap \{T=t_n\} \cap \{B^{(t_n)}\in A\}
		\bigr) \\
	&=	\sum_{n\ge1} \Pr\bigl( E \cap \{T=t_n\} \bigr)
			\Pr(B^{(t_n)}\in A) \\
	&=	\Pr(E) \Pr( B \in A )
	\text.
	\qedhere
\end	{align*}
\end	{proof}

\begin	{theorem}
Let $T$ be an almost surely finite stopping time.
Define \term{reflected} Brownian motion by \[
	R^T_t =
	\begin	{cases}
		B_t	& \text{if \( t \le T \),} \\
		2B_T - B_t	& \text{if \( t \ge T \).}
	\end	{cases}
\]
Then $R^T_t$ is Brownian motion.
\end	{theorem}
\begin	{proof}
Define \(B_{t \le T} = B_t\) if \( t \le T \) and \( = B_T \) otherwise.
The operation of concatenation at~$t$ is continuous \[
	C[0,\infty) \times [0,\infty) \times C[0,\infty) \to C[0,\infty)
	\text,
\] hence measurable,
and sends \( (B_{t \le T},T,B^{(T)}) \mapsto B \)
and \( (B_{t \le T},T,-B^{(T)}) \mapsto R^T \).
But $\pm B^{(T)}$ are independent of both $T$ and~$B_{t \le T}$,
and have the same distribution as~$B$.
\end	{proof}

\begin	{proposition}
Let $x$,~\( c \ge 0 \).
Then \( \Pr(\, T_x \le t \), \( B_t \le x - c \,)
	= \Pr( B_t \ge x+c ) \).
\end	{proposition}
\begin	{proof}
Reflecting at the first hitting time~$T_x$ of~$x$
gives a distribution-preserving bijection \( B_t \leftrightarrow R^{T_x}_t \)
under which correspond these two events.
\end	{proof}

\begin	{theorem}
\begin	{equation*}
	\Pr(T_x \le t) = \Pr\bigl( \abs{B_t} \ge \abs x \bigr) \text.
\end	{equation*}
\end	{theorem}
\begin	{proof}
For \( x \ge 0 \),
using the previous proposition and that \( \Pr(B_t=x) = 0 \),
\begin	{align*}
	\Pr( T_x \le t)
	&=	\Pr(\, \text{\( T_x \le t \), \( B_t \le x \)} \,)
	+	\Pr(\, \text{\( T_x \le t \), \( B_t >   x \)} \,) \\
	&=	2\Pr( B_t \ge x ) \text.
	\qedhere
\end	{align*}
\end	{proof}

\begin	{theorem}
Let \( 0 \le t_0 \le t_1\).
Then \[
	\Pr(0 \in B_{[t_0,t_1]}) = \tfrac2\pi\arccos\sqrt{t_0/t_1} \text.
\]
\end	{theorem}
\begin	{proof}
\begin	{align*}
	\Pr(0 \in B_{[t_0,t_1]})
	&=	2\int_{B_{t_0}\ge0}
			\Pr(\, 0 \in B_{[t_0,t_1]} \wcond B_{t_0} \,)
		\,dB_{t_0} \\
	&=	2\int_{x\ge0} \Pr( T_{-x} \le t_1-t_2 )
			\frac{dB_{t_0}}{dx} \,dx \\
	&=	\frac1{\pi\sqrt{t_0}}
		\int_{\substack{x\ge0\\t\le t_1-t_0}}
		\frac {\exp\bigl(-\tfrac12x^2(1/t+1/t_0)\bigr)}{t^{3/2}}
		\,dx \\
	&=	\tfrac2\pi\arctan\sqrt{t_1/t_0-1} \text.
	\qedhere
\end	{align*}
\end	{proof}
