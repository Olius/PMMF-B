\chapter	{Stochastic Processes and Brownian Motion}

\section	{Stochastic Processes}

\subsection	{The Kolmogorov Extension Theorem for Stochastic Processes}

\begin	{theorem}
\label	{thm:kolmogorov-extension-for-processes}
Let $S$ be a Polish topological space,
and $\mathcal T$ a set the elements of which we shall refer to as \term{times}.
For every finite collection \( I \subseteq \mathcal T \) of times,
let \( P_I \) be a probability measure on~$S^I$.
Suppose furthermore that for any finite sets of times \( I \subseteq J \),
denoting \( s_{J \supseteq I} \colon S^J \to S^I \) the projection,
we have \( s_{I \subseteq J}(P_J) = P_I \).
Then there exists a unique probability measure~$P$ on~$S^{\mathcal T}$
such that, with \( s_I \colon S^{\mathcal T} \to S^I \)
the projection, \( s_I(P) = P_I \).
\end	{theorem}
\begin	{proof}
This is a special case of general Kolmogorov extension,
\autoref{thm:kolmogorov-extension}:
let \( S_I = S^I \), with the projections~$s_{J \supseteq I}$
induced by the inclusions \( I \subseteq J \).
It is then easy to see that the projective limit of this system
is the product~$S^{\mathcal T}$,
and by the axiom of choice the projections~$s_I$ are surjective.
In addition, since countable products of Polish spaces are Polish,
and a measure on a Polish space is always tight,
all measures~$P_I$ are automatically tight.
Consequently, \autoref{thm:kolmogorov-extension} applies.
\end	{proof}


\subsection	{Continuity and Kolmogorov's Theorem}

\begin	{proposition}
Let \( \mathcal S \subseteq \mathcal T \) be a set of times.
Then for \( A \in \sgen(\, s_t : t \in \mathcal S \,) \):
\begin	{enumerate}
\item	the assertion \( \omega \in A \) depends only on the values~$\omega(t)$
	at times \( t \in \mathcal S \);
\item	there exists a \emph{countable} \( \mathcal N \subseteq \mathcal S \)
	such that \( A \in \sgen(\, s_t : t \in \mathcal N \,) \).
\end	{enumerate}
\end	{proposition}
\begin	{proof}
For either property, the set of events which satisfy it
is a \salg\ which contains $\sgen(s_t)$ for every \( t \in \mathcal S \),
and so also every event of~\( \sgen(\, s_t : t \in \mathcal S \,) \).
\end	{proof}

\begin	{corollary}
If $\mathcal T$ is a topological space
in which every neighborhood of some \( t_0 \in \mathcal T \) is uncountable,
and $S$ does not carry the trivial topology,
then continuity is not measurable in~$S^{\mathcal T}$
by~\( \sgen(\, s_t : t \in \mathcal T \,) \).
\end	{corollary}


\subsubsection
	{Borel for compact convergence implies pointwise measurable}

\begin	{theorem}
Let $X,Y$ be second countable spaces,
with $X$ locally compact and $Y$ regular.
Equip the space~$C(X,Y)$ of continuous functions \( X \to Y \)
with the compact-open topology,
and denote by~$\mathcal B$ the induced Borel \salg.
Then \( \mathcal B = \sgen(\, s_x : x \in X \,) \).
\end	{theorem}

\begin	{corollary}
In the space of continuous paths \( I \to S \),
where \( I \subseteq \R \) is an interval and $S$ is separable metric,
the Borel \salg\ induced by the topology of uniform convergence on compact sets
coincides with the product \salg.
\end	{corollary}

\begin	{proof}
Under these hypotheses,
the compact-open topology on~$C(X,Y)$ admits a countable basis
of sets of the form \( R(K,V) = \{\, f(K) \subseteq V \,\} \),
where \( K \subseteq X \) is compact and \( V \subseteq Y \) is open.
It thus suffices to show that \( R(K,V) \in \sgen(\, s_x : x \in X \,) \).
Since $Y$ is regular,
\( V = \bigcup_n V_n = \bigcup_n \overline{V_n} \)
is a countable union of opens whose closures are contained in~$V$.
We can suppose \( V_n \subseteq V_{n+1} \).
Thus \( R(K,V) = \bigcup_n R(K,V_n) = \bigcup_n R(K,\overline{V_n}) \).
Finally, for \( F \subseteq Y \) closed,
\( R(K,F) = R(Q,F) = \bigcap_{q \in Q} R(q,F) \)
where \( Q \subseteq K \) is a countable dense subset.
\end	{proof}
