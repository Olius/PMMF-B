\chapter	{Kolmogorov Extension Theorem}

\begin	{theorem}	[Kolmogorov extension]
Let \( s_{J \ge I} \colon S_J \to S_I \)
be a projective system of Hausdorff spaces,
with limit~$S$.
Suppose the projections \( s_I \colon S \to S_I \) are surjective.\footnote
	{Note that this obviates the need to consider general
	filtered limits,
	since it would imply that parallel arrows \( J \to I \)
	must be sent to the same function \( S_J \to S_I \).}
For each~$I$, let $P_I$ be a tight probability measure on~$S_I$.
If \( s_{I \le J}(P_J) = P_I \) whenever \( I \le J \),
then there exists a unique probability measure~$P$ on~$S$ such that
\( s_I(P) = P_I \) for all~$I$.
\end	{theorem}
The proof is an application of the Carathéodory extension theorem.
Indeed, the product \sig algebra on~$S$ is generated by the~$s_I$,
and surjectivity of the~$s_I$ allows us to define a finitely additive
probability on the product algebra generated by the~$s_I$.
The work lies in showing \sig additivity.

First note that the product algebra generated by the~$s_I$
is simply the union
\( \bigcup_I \sgen(s_I) \),
as this union already contains finite conjunctions of its events
due to the system being directed.
To wit, let \( A_k \subseteq S_{I_k} \) be a finite collection of events.
Since the indices are directed there is an \( I_0 \ge I_k \) for all~$k$.
Then as \( s_{I_k} = s_{I_k \le I_0} s_{I_0} \),
\begin	{equation}
\label	{eq:product-algebra-is-union}
	\bigcap_k s_{I_k}^* A_k
		= \bigcap_k s_{I_0}^* s_{I_0 \ge I_k}^* A_k
		= s_{I_0}^* \bigcap_k s_{I_0 \ge I_k}^* A_k
		\in \sgen(s_{I_0})
		\subseteq \bigcup_I \sgen(s_I)
	\text.
\end	{equation}

Now, surjectivity of~$s_I$ implies injectivity of~$s_I^*$.
Therefore the union in~\eqref{eq:product-algebra-is-union}
is a directed union of isomorphic copies of the~$\sgen(P_I)$.
Moreover, by hypothesis \( P_J(s_{J \ge I}^*A_I) = P_I(A_I) \),
so that \( P(s_I^*A_I) = P_I(A_I) \) is a valid definition
of a finitely additive probability~$P$ on the union.

To apply the Carathéodory extension theorem,
we must check \sig additivity of~$P$ on the product algebra,
which is equivalent to continuity in~$\emptyset$.
So let \( A_n \subseteq S_{I_n} \) be a sequence of events
such that \( s_{I_n}^* A_n \subseteq S \) is decreasing
and such that \( P_{I_n}(A_n) \ge \varepsilon > 0 \) for all~$n$.
We will show that \( \bigcap_n A_n \ne \emptyset \).
Iteratively replacing~$I_{n+1}$ by an upper bound of $I_n$~and~$I_{n+1}$,
we can suppose the~$I_n$ increasing.
By tightness of the measures, there are compact \( C_n \subseteq A_n \)
rendering \( P_n(A_n \setminus C_n) \) so small that
\( \sum_{k \le n} P_k(A_k \setminus C_k) < \varepsilon/2 \).
Set \( K_n = \bigcap_{k \le n} s_{I_n \ge I_k}^* C_k \).
Since \( s_{I_n = I_n}^* C_n = C_n \) is compact
and \( \bigcap_{k < n} s_{I_n \ge I_k}^* C_k \subseteq S_{I_n} \) is closed,
\( K_n \subseteq S_{I_n} \) is compact.
By construction, \( s_{I_n \le I_{n+1}}(K_{n+1}) \subseteq K_n \).
Finally, by the choice of~$C_n$,
\(	P_n( A_n \setminus K_n )
\le	\sum_{k \le n} P_k( A_k \setminus C_k )
<	\varepsilon/2
\), so that \( P_n(K_n) \ge \varepsilon/2 > 0 \),
and \( K_n \ne \emptyset \).
Thus restricting \( s_{I_{n+1} \ge I_n} \colon K_{n+1} \to K_n \)
yields a projective system of nonempty compact Hausdorff spaces.

\begin	{lemma}
A filtered limit of nonempty compact Hausdorff spaces
is also nonempty compact Hausdorff.
\end	{lemma}

Hence \( \emptyset \ne \lim_n K_n \subseteq \lim_n A_n \).
As the projections \( S \to S_{I_n} \) are surjective,
so is the induced map \( S \to \lim_n S_{I_n} \).
The pullback along this map of the nonempty subset
\( \lim_n A_n \subseteq \lim_n S_{I_n} \)
is precisely \( \bigcap_n s_{I_n}^* A_n \),
and is therefore nonempty.
\qed
