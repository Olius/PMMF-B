\section	{Brownian Motion as a Markov Process}

\begin	{proposition}
Brownian motion~$B_t$ in $d$~dimensions
is a Markov process with translation-invariant
stationary transition probability given at the origin by
\[ dp^t = \frac {\exp(-\norm x^2/2t)} {(2\pi t)^{d/2}} \,dx \text. \]
\end	{proposition}
\begin	{proof}
\begin	{align*}
	\Pr(\, B_t \in A \wcond B_{r\le s} \,)
	&= \Pr(\, B_s + B_t - B_s \in A \wcond B_{r\le s} \,) \\
	&= \Pr\bigl(f(B_s,B_t-B_s)\big\cond B_{r\le s}\bigr)
\end	{align*}
if we define the bounded measurable
\( f(x,y) = [x+y\in A] \).
Proceeding then by use of the monotone class theorem,
since \( B_t-B_s \) is independent of~$B_{r\le s}$
and distributed like~$B_{t-s}$,
\[	\Pr(\, B_t \in A \wcond B_{r\le s} \,)(\gamma)
=	\Pr\bigl(f(\gamma(s),B_{t-s})\bigr),
\]
which, being a function only of \( \gamma(s) = B_s \),
is the transition probability \( \Pr(\,B_t\in A\wcond B_s\,) \),
and evidently depends only on the difference \( t-s \).
\end	{proof}
