\section	{Solving the Heat Equation}

\begin	{proposition}
If $S$~and~$T$ are commuting Markovian semigroups,
with generators $A$~and~$B$ respectively,
then the product Markovian semigroup \( t \mapsto S_t T_t \)
has generator \( A + B \).
\end	{proposition}
\begin	{proof}
This is a consequence of the product rule applied to composition,
which itself is a consequence of the chain rule:
indeed, if $[x,y]$ is a bilinear map
then it is continuously Fréchet-differentiable with
\( d[x,y] = [dx,y] + [x,dy] \),
and since composition is bilinear and \( dS = dT = 1 \) at \( t = 0 \),
\[ d(ST) = dS + dT \quad \text{at \( t = 0 \).} \qedhere \].
\end	{proof}

\begin	{corollary}
If two independent stationary Markov processes $X$~and~$Y$
have transition probabilities $p$~and~$q$
and generators $A$~and~$B$ respectively,
then $(X,Y)$ has generator \( A \oplus B \).
\end	{corollary}
\begin	{proof}
Apply the monotone convergence theorem
to show that the transition probabilities of~$(X,Y)$
are the product of those of $X$ and~$Y$:
ultimately this translates to
\( d(X_t,Y_t\cond X_0,Y_0) = d(X_t\cond X_0) \, d(Y_t\cond Y_0) \)
pointwise almost surely,
valid by independence of $X$~and~$Y$.
Then the conclusion is immediate from the previous proposition
since the Markovian semigroup of~$(X,Y)$ is the direct sum
of those of $X$ and~$Y$.
\end	{proof}

\begin	{definition}
\term{Space-time Brownian motion} started at~$(t_0,x_0)$
is defined as the semi-deterministic process \( (t_0-t,x_0+B_t) \).
By the previous corollary it has generator
\[ -\frac{\partial}{\partial t} + \frac12 \Lap \text. \]
\end	{definition}

\begin	{theorem}
Let \( D \subseteq \R^d \) be regular,
\( h \colon \partial \bigl( [0,\infty) \times D \bigr) \to \R^d \)
continuous and bounded for every fixed \( t \in [0,\infty) \).
Define $\tau$ to be the time of first exit
of space-time Brownian motion~$H$ from \( D \times [0,t] \).
Then \( u = \Pr[\, h(H_\tau) \wcond H_0 = (t_0,x_0) \,] \)
is a solution to the dynamic heat equation
\begin	{equation}
\label	{eq:dynamic-heat}
\begin	{cases}
	\frac{\partial u}{\partial t} = \frac12 \Lap u
		& \text{if \( x \in D \) and \( t > 0 \),} \\
	u = h & \text{if \( x \in \partial D \) or \( t = 0 \).}
\end	{cases}
\end	{equation}
\end	{theorem}
