\section	{The Laplace Equation}

\begin	{definition}
Let \( \tau_x = \inf\{\, t>0 \mid x+B_t \notin D \,\} \)
be the time of first exit of~$B$ from~$D$.
By the 0--1~law, \( \Pr(\tau_x = 0) = 0 \)~or~$1$;
call \( x \in \partial D \) \term{regular} if~$1$.
If all \( x \in \partial D \) are regular then $D$ is \term{regular}.
\end	{definition}

\begin	{theorem}[Poincaré--Zaremba]
If there is a finite-length cone at \( x \in \partial D \)
contained in~$\co D$,
then $x$ is regular.
\end	{theorem}
\begin	{proof}
Writing $T_C$ for the first entry time
in the corresponding infinite-length cone~$C$,
by rescaling invariance
\( \Pr(\tau_x=0) \ge \Pr(T_C=0) \ge \Pr(B_1 \in C) > 0 \).
\end	{proof}

\begin	{lemma}
The function \( \Pr(\tau_x > t) \) is upper-semicontinuous in \( x \in D \)
for every \( t > 0 \).
\end	{lemma}
\begin	{proof}
\begin	{equation*}
\Pr(\tau_x > t)
%&=	\lim_{s\to0} \Pr( x+B_{[s,t]} \subseteq D ) \\
=	\lim_{s\to0}
	\Pr\bigl[ \Pr(\, x+B_{[s,t]} \subseteq D \wcond B_s \,) \bigr]
\text. \qedhere
\end	{equation*}
\end	{proof}

\begin	{corollary}
For every \( t > 0 \) and regular \( x_0 \in \partial D \),
\( \lim_{x \to x_0} \Pr(\tau_x > t) = 0 \text. \)
\end	{corollary}

\begin	{theorem}
The harmonic function defined in~\eqref{eq:laplace-solution}
is continuous at every regular \( x_0 \in \overline D \).
\end	{theorem}
\begin	{proof}
Define \( \Delta_x = h(B_{\tau_x}) - h(x_0) \).
By continuity at \( x_0 \in \partial D \),
we can suppose
\( \abs{x-x_0} < 2\delta \) implies \( \abs{h(x) - h(x_0)} < \epsilon \).
Let $T_x$ be the first exit time of \( x + B_t \) from~$B(x_0,2\delta)$.
Then we have
\begin	{align*}
\Pr\bigl(\abs{\Delta_x}\bigr)
&=	\Pr\bigl( [\tau_x <   T_x] \abs{\Delta_x} \bigr)
+	\Pr\bigl( [\tau_x \ge T_x] \abs{\Delta_x} \bigr) \\
&\le	\epsilon
+	2\norm h_\infty \Pr(\tau_x \ge T_x)
\text.
\end	{align*}
Furthermore, for any \( t > 0 \), if we suppose \( \abs{x-x_0} < \delta \),
\begin	{align*}
\Pr(T_x \le \tau_x)
&\le	\Pr(T_x \le t) + \Pr(\tau_x > t) \\
&\le	\Pr\bigl(\max_{0 \le s \le t} \norm{B_s} \ge \delta\bigr)
+	\Pr(\tau_x > t)
\text.
\end	{align*}
Taking the limit superior as \( x \to x_0 \) cancels the right summand,
and then the limit as \( t \to 0 \) the left.
\end	{proof}

\begin	{theorem}
If $D$ is regular and \( \tau_x < \infty \) almost surely
for every \( x \in D \),
and \( h \colon \partial D \to \R \) is bounded and continuous,
then \eqref{eq:laplace-solution} gives the unique solution
\( f \in \Cont^2(D) \cap \Cont(\overline D) \)
of the steady-state heat equation
\begin	{equation}
\label	{eq:heat-equation}
	\begin	{cases}
	\Lap f = 0  &\text{in~$D$,} \\
	f = h	    &\text{on~$\partial D$.}
	\end	{cases}
\end	{equation}
\end	{theorem}
\begin	{proof}
The previous results show that $f$ is indeed a solution;
uniqueness follows from the maximum principle
since the difference of two solutions is harmonic in~$D$
and zero on~$\partial D$.
\end	{proof}
