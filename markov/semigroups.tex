\section	{Markovian Semigroups and Infinitesimal Generators}

\begin	{definition}
Let $\mb(S)$ be the Banach space of bounded measurable functions on~$S$
with the supremum norm,
and let \( B \subseteq \mb(S) \) be a closed subspace.
An operator \( T \colon B \to B \) is \term{Markovian}
if it is positivity-preserving, of norm~$1$,
and \( \sup_{f\le1} Tf = 1 \) pointwise.
Put on the monoid of Markovian operators
the topology of pointwise convergence.
A \term{Markovian semigroup} on~$B$ is a continuous monoid homomorphism
of~$[0,\infty)$ into the monoid of Markovian operators on~$B$.
\end	{definition}

\begin	{lemma}
Let $M$ be a monoid with the structure of a uniform space,
in which multiplication is uniformly continuous in both arguments.
Then continuity at \( 0 \in [0,\infty) \)
is enough for continuity of a monoid homomorphism
\( [0,\infty) \to M \).
\end	{lemma}
\begin	{proof}
\( m_t = m_{t-s} m_s \) if \( s \le t \).
\end	{proof}

\begin	{proposition}
Restricting a Markovian semigroup~$T_t$ on~$B$ which is not continuous
to the set of \( f \in B \) at which it is continuous at \( t = 0 \)
yields a closed subspace \( B_0 \subseteq B \)
on which $T_t$ is a continuous Markovian semigroup.
\end	{proposition}

\begin	{definition}
Given a Markovian semigroup~$T_t$ on~$B$,
its \term{infinitesimal generator}~$A$ is defined over its natural domain as \[
	\text{\(Af = \lim_{t\to0} t^{-1}(T_t f - f)\) in~$B$.}
\]
\end	{definition}

\begin	{theorem}
The domain of~$A$ is a dense subset of~$B$ stable under~$T_t$,
over which $A$~and~$T_t$ commute,
and \( f(t,x) = (T_t f_0)(x) \) solves the Cauchy problem \[
	\begin	{cases}
	\frac\partial{\partial t} f = Af & t \ge 0 \text,\\
	f = f_0                          & t =   0 \text.
	\end	{cases}
\]
\end	{theorem}

\begin	{theorem}
If
\[ \lim_{\delta\to0} \alpha_r(\delta) = 0 \quad \text{for all \(r>0\),} \]
then \( \Cont_0(S) \subseteq B_0 \).
If $S$ is compact then the converse holds as well.
\end	{theorem}

\begin	{definition}
A \term{Fellerian semigroup} is a Markovian semigroup on~$\Cont_0(S)$.
A stationary Markov process is \term{Fellerian}
if its Markovian semigroup restricts to a Fellerian semigroup.
\end	{definition}

\begin	{theorem}
For a semigroup on~$\mb(S)$ arising from a Markov process
as
\begin	{equation}
\label	{eq:process-semigroup-correspondence}
	(T_tf)(x) = \Pr[\, f(X_t) \wcond X_0 = x \,]
\end	{equation}
to be Fellerian,
it is enough to restrict to~$\Cont_0(S)$;
in other words, for this kind of semigroup,
continuity on~$\Cont_0(S)$ is a consequence of its preservation.
\end	{theorem}

\begin	{theorem}
If $S$ is locally compact,
there is a one-to-one correspondence
between Feller processes and Feller semigroups
given by~\eqref{eq:process-semigroup-correspondence}.
\end	{theorem}
